%
% $Id: slides.tex 8 2014-02-04 21:01:21Z nicb $
%
% Copyright (C) 2007-2009 Nicola Bernardini nicb@sme-ccppd.org
% 
% This work is licensed under a Creative Commons License, and specifically the
% 
%   Creative Commons Attribution-ShareAlike 2.5 License
%   http://creativecommons.org/licenses/by-sa/2.5/legalcode
% 
% Check http://www.creativecommons.org/ for more information on
% Creative Commons Licenses and the Creative Commons Project.
%
% Set the macros below to whatever is appropriate in a given context
%
\newcommand{\argroot}{..}
\newcommand{\beamerslides}{\argroot}
\newcommand{\imagedir}{\argroot/media}
\newcommand{\exampledir}{\argroot/media}
\newcommand{\templatedir}{\argroot/templates/sme-ccppd}
\newcommand{\datadir}{../media}
\newcommand{\LLdE}{\emph{La L\'egende d'Eer}\xspace}
\documentclass[mode=\printmode,compress,xcolor=dvipsnames]{beamer}
\usepackage{ifthen}

\usepackage{beamerthemeSME-CCPPD}

% \ifthenelse{\equal{\printmode}{handout}}%
% {%
% 	\usepackage{pgfpages}
% 	\pgfpagesuselayout{1 on 1}[a4paper,landscape,border shrink=10mm]
% }{}

\usepackage{colortbl}
\usepackage[italian]{babel}
\usepackage{pgf}
\usepackage{xspace}

\usepackage{multimedia}
\usepackage{xmpmulti}
\usepackage{hyperref}
\usepackage{gitinfo2}
\usepackage{gensymb}
\usepackage{hhline}


\newcommand{\cpyear}{2019}
\newcommand{\cpholder}{Nicola Bernardini}
\newcommand{\cpholderemail}{nicola.bernardini@conservatoriopollini.it}

% Use some nice templates

%\beamertemplateshadingbackground{red!10}{structure!10}
\beamertemplatetransparentcovereddynamic
\beamertemplateballitem
\beamertemplatenumberedballsectiontoc

% My colors
\definecolor{notdone}{gray}{0.35}

%\usecolortheme[named=MyColor]{structure}
%\usecolortheme[named=MyColor]{structure}
\beamertemplateshadingbackground{white!10}{white!10}

%
% Author: Nicola Bernardini <nicb@sme-ccppd.org>
%
% Copyright (c) 2004 Nicola Bernardini
% Copyright (c) 2004 Conservatorio "C.Pollini", Padova
%
% This work is licensed under the Creative Commons 
% Attribution-ShareAlike License. To 
% view a copy of this license, visit 
% http://creativecommons.org/licenses/by-sa/2.0/ 
% or send a letter to Creative Commons, 
% 559 Nathan Abbott Way, Stanford, California 94305, USA.
%
% Some rights reserved.
% CVSId : $Id: macros.tex 7 2014-02-04 18:16:40Z nicb $
%
\newcommand{\rcstag}{rel.\gitAbbrevHash\xspace}
\newcommand{\hhref}[1]{\href{#1}{#1}\xspace}
\newcommand{\sharedimages}{\templatedir}
\pgfdeclareimage[width=0.45cm]{speaker}{\sharedimages/speaker}
\newcommand{\listento}[2] % \listento without overlay
{%
  \href{run:#1}%
  {%
	  \hspace{0.05em}\raisebox{-0.2\baselineskip}{\pgfuseimage<.->{speaker}}%
    \hspace{0.3em}#2%
  }\xspace
}
\newcommand{\slideinput}[1]{\input{\beamerslides/#1}}
\newcommand{\Ts}{\rule{0pt}{2.6ex}}%
\newcommand{\Bs}{\rule[-1.2ex]{0pt}{0pt}}%


\title[MEeSN \gitAbbrevHash]
{\small%
  Musica Elettronica e Suggestioni Numeriche\\
	{\tiny (\rcstag)}
}

\author{%
  Nicola Bernardini\\
	{\tiny \href{mailto:\cpholderemail}{\cpholderemail}}
}
% \institute[SME-CCPPD]%
% {%
% 	\href{http://www.conservatoriopollini.it}
% 		 {Conservatorio ``C.Pollini'' -- Padova}
% }
\date[Padova 18/03/2019]{Musica e/\`e Scienza - Auditorium Pollini, 18 marzo 2019}

\begin{document}
\newcounter{ms}
  
%%%% START %%%%

\begin{frame}
	\titlepage
\end{frame}
  
\section{Premessa e Digressione}

\begin{frame}
   \frametitle<+->{Piccola premessa (e digressione)}

   \begin{Large}
     \uncover<+->{Di che musica parleremo?}\\[\baselineskip]
     \uncover<+->{I generi musicali non esistono.}\\[\baselineskip]
     \uncover<+->{Esistono per\`o le \emph{FUNZIONI} musicali}
   \end{Large}

\end{frame}

\begin{frame}
   \frametitle<+->{Funzioni Musicali}

   \uncover<+->{Rito}\\[\baselineskip]
   \uncover<+->{Intrattenimento}\\[\baselineskip]
   \begin{Large}
      \uncover<+->{\emph{SPECULAZIONE INTELLETTUALE}}
   \end{Large}
\end{frame}


\section{Musica e Numero}

\begin{frame}
   \frametitle<+->{La fascinazione numerica}

   \uncover<+->{I suoni sono ``pieni'' di numeri}\\[\baselineskip]
   \uncover<+->{I numeri ``cantano''}\\[\baselineskip]
   \uncover<+->{\listento{\exampledir/Nozze_aria1-frammento.ogg}{}}


\end{frame}

\begin{frame}
   \frametitle<+->{I numeri nella composizione}

   \uncover<+->{Molta ``tecnologia numerica'' semplice (permutazione, inversione,
   retrogradazione, ecc.) \`e da sempre utilizzata in musica}\\[\baselineskip]
   \uncover<+->{\listento{\exampledir/Berio_Requies-frammento.ogg}{Esempio: Luciano Berio - \emph{Requies} (1984)}}

\end{frame}

\section{Musica e/\`e Scienza}

\begin{frame}
   \frametitle<+->{Musica \emph{\`e} Scienza?}

   \begin{itemize}[<+->]
      \item La Musica \`e un'attivit\`a artistica:
        \begin{itemize}[<+->]
           \item completamente \emph{arbitraria}
           \item guidata solo da criteri estetici
        \end{itemize}
     \item Una ``scienza'' della composizione musicale non esiste
   \end{itemize}

   % perch\'e insistiamo nel unire musica e scienza?

\end{frame}

\section{La Musica Elettronica}

\begin{frame}
   \frametitle<+->{La Musica Elettronica}

   \begin{itemize}[<+->]

      \item L'introduzione delle tecnologie elettroniche (prima analogiche poi
              digitali) in musica cambia sostanzialmente i paradigmi
              compositivi

      \item I compositori affrontano dimensioni parametriche continue

      \item La Matematica diventa uno strumento utile per la composizione

   \end{itemize}

   % disclaimer: a scelta strumentale delle ``tecnologie matematiche'' rimane
   % comunque arbitraria;
   % bella matematica per brutta musica e viceversa

\end{frame}

\section{La L\'egende d'Eer}

\begin{frame}
        \frametitle<+->{Iannis Xenakis -- 1922--2001}

   \begin{itemize}[<+->]

      \item Xenakis combina una tripla formazione: ingegnere, architetto,
            compositore

     \item fascinazione dall'approccio scientifico all'investigazione della
             natura 

     \item utilizzo estensivo di funzioni di distribuzione statistica

     % mettere in contesto: dodecafonia -> serialismo integrale, aleatoriet\`a
     % Cageana, approccio di Xenakis
   \end{itemize}

\end{frame}

\begin{frame}
   \frametitle<+->{\LLdE (1977-1978)}

   \begin{itemize}[<+->]

      \item Commissionato per l'inaugurazione del \emph{Centre Georges Pompidou}

      \item Parte di un progetto pi\`u ampio di \emph{gesamtkunstwerk}, il
              \emph{Diatope} (una struttura architettonica con 1680 fari, 400
              specchi e 4 laser)

      \item Immaginato come un'immersione del pubblico in un oceano di suoni


   \end{itemize}

\end{frame}

{\Large Funzioni Musicali}

\begin{itemize}

        \item Speculazione: funzione primaria riconosciuta dai Greci (cit. Platone
              Repubblica), Boezio e Cassiodoro -> Quadrivium: aritmetica (numeri),
              geometria (numeri nello spazio), musica (numeri nel tempo), astronomia
              (numeri nello spazio e nel tempo)
        \item musica theoretica e musica practica (scontro tra i teorici,
                Vincenzo Galilei, i fratelli Monteverdi, ecc.)
        \item importanza della riproducibilit\`a: tradizione orale favorisce l'esecuzione, la scrittura favorisce la costruzione
                architettonica: Beethoven e seguenti; la riproducibilit\`a dei
                suoni restaura la supremazia dell'esecuzione (e stabilisce un
                mercato)


\end{itemize}


\end{document}
