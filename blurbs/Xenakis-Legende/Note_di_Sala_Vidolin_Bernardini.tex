%
% Note di sala per il concerto del 18 marzo 2018
%
% Redatte da Nicola Bernardini sulla base di annotazioni
% di Alvise Vidolin.
%
%
\documentclass{scrartcl}
\usepackage{xspace}
\usepackage[italian]{babel}

\newcommand{\LDE}{\emph{La L\'egende d'Eer}\xspace}
\title{Iannis Xenakis -- \LDE\\{\small note di sala a cura di Alvise Vidolin e Nicola Bernardini}}
\date{Auditorium Pollini, 18 marzo 2019, ore 18:30}

\begin{document}

\maketitle

\LDE \`e un lavoro per soli altoparlanti (acusmatico)
composto da Iannis Xenakis tra il 1977 e il 1978 per l'inaugurazione del
\emph{Centre Georges Pompidou} e ivi presentato nel 1978 all'interno di uno
spazio concepito dallo stesso compositore (che era anche ingegnere e
architetto), il \emph{Diatope}, in un \emph{gesamtkunstwerk} che includeva una
diffusione spazializzata e un sofisticato sistema di illuminazione composto da
1680 fari, 400 specchi e 4 laser.

Xenakis trasse ispirazione per la composizione di \LDE da cinque fonti
letterarie: il mito di \emph{Er} descritto nelle ultime pagine della
\emph{Repubblica} di Platone, mito del soldato morto improvvisamente ritornato
in vita a raccontare ci\`o che succede nell'aldil\`a;
un piccolo estratto delle \emph{Pens\'ees} di Blaise Pascal,
dedicato all'insignificanza dell'essere umano di fronte all'infinito e all'immensit\`a della natura;
il \emph{Pimandro} del misterioso Ermete Trismegisto, racconto sulla
conoscenza della realt\`a;
un testo dell'autore romantico tedesco Johann Paul Richter, \emph{Sibenk\"as},
che descrive un rivelazione onirica di Cristo che trova l'universo non
governato da alcun dio;
ed infine un saggio scientifico di Robert Kischer che espone una teoria sulla
formazione delle stelle.

\LDE, gigantesco affresco sonoro di 45 minuti, \`e poi rimasto come lavoro musicale a se stante
che immerge gli ascoltatori in un universo sonoro estremamente elaborato che
si sviluppa in una archetipica forma circolare che pone l'ascolto sulle soglie
dell'infinit\`a e dell'immenso.

Viene qui riproposto in una rilettura spaziale di Alvise Vidolin e Nicola Bernardini concepita
appositamente per l'Auditorium Pollini.
\end{document}
